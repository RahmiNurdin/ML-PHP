\section{PHP sebagai Bahasa Pemprograman}
\subsection{Mengenal PHP sebagai Bahasa Pemprograman}
PHP adalah singkatan dari "PHP: Hypertext Prepocessor", yaitu bahasa pemrograman yang digunakan secara luas untuk penanganan pembuatan dan pengembangan sebuah situs web dan bisa digunakan bersamaan dengan HTML. PHP diciptakan oleh Rasmus Lerdorf pertama kali tahun 1994. Sejak versi 3.0, nama bahasa ini diubah menjadi "PHP: Hypertext Prepocessor" dengan singkatannya "PHP". PHP versi terbaru adalah versi ke-5.
Bahasa pemrograman ini termasuk ke dalam bahasa pemrograman yang serba guna dan mendukung terhadap PHP Code, Text, HTML, CSS, dan JavaScript. Bahasa pemrograman PHP juga mampu menangani banyak hal dalam pengembangan web.
Bahasa ini mampu mengumpulkan data serta membuat konten laman web menjadi lebih dinamis. Bahasa ini dapat digunakan untuk membuat, membuka, membaca, menulis, dan menutup file yang berada di sisi server. Bahasa pemrograman ini juga dapat menangani database, seperti menghapus, menambah, atau memodifikasi data.
Tidak hanya itu, bahasa ini juga dapat menangani keamanan dari data. Bahasa pemrograman ini dapat digunakan untuk membatasi pengguna untuk mengakses beberapa laman pada website yang dikembangkan. Bahasa pemrograman ini juga mampu mengenkripsi data yang ada.
\subsection{Sejarah PHP}
Pada awalnya PHP dikenal dengan singkatan Personal Home Page. Karena server tersebut di peruntukan untuk website pribadi. Tetapi untuk saat ini PHP sudah bermetamorfosis menjadi bahasa pemrograman yang sangat populer yang digunakan untuk website terkenal seperti Wikipedia,wordpress,joomla,dll.
Untuk saat ini php dikenal dengan singkatan Hypertext Preprocessor sebuah kepanjangan rekursif, yakni permainan kata dimana kepanjangannya terdiri dari singkatan itu sendiri. Bahasa pemrograman php banyak digunakan karena sifatnya yang open source yaitu dapat digunakan secara gratis.Bahasa Pemrograman PHP adalah bahasa pemrograman script server-side yang didesain untuk pengembangan web. Selain itu, PHP juga bisa digunakan sebagai bahasa pemrograman umum. Pertama kali di kembangkan oleh Rasmust Lerdorf pada tahun 1995, dan sekarang php dikembangkan oleh The PHP Group.
Sementara untuk penyisipan kode php dapat disisipkan pada html. Karena php bersifat Scripting Language atau Bahasa Pemprograman script. PHP sendiri memiliki perkembangan versi dari tahun ketahun di antaranya :
  \begin{enumerate}
     \item PHP/ FI : Personal Home Page / Forms Interfreter.
      Berasal dari tahun 1994 yang dikembangkan oleh Rasmus Lerdoft untuk membuat kode program (script) dengan Bahasa perl untuk web pribadinya. Salah satu kegunaan script ini adalah untuk menampilkan resume pribadi dan mencatat jumlah pengunjung ke sejumlah website.
     \item PHP/ FI : Personal Home Page / Form Interpreter 2
      Pada 1996 Rasmus Lerdoft mengumumkan PHP/FI versi 2.0. versi 2 ini dirancang lerdoft pada saat mengerjakan sebuah proyek di University of Toronto yang membutuhkan pengolahan data dan tampilan web yang rumit.
     \item PHP : Hypertext Prepocessor 3
      Terjadi pada pertengahan tahun 1997, telah banyak menarik perhatian programmer namun Bahasa ini memiliki masalah dengan kestabilan yang kurang bisa diandalkan.
     \item PHP : Hypertext Preprocessor 4
      Dalam fitur ini PHP memperkenalkan beberapa fitur lanjutan, seperti layer abstraksi antara PHP dan web server, menambahkan mekanisme thread-safety, dan two-stage parsing.
    \item PHP : Hypertext Preprocessor 5
      Versi PHP terakhir hingga saat ini, yaitu PHP 5.X diluncurkan pada 13 juli 2004. PHP 5 telah mendukung penuh pemrograman object dan peningkatan perfoma melalui Zend engine versi 2.
    \item PHP Hypertext Preprocessor 7
      Pada versi ini programmer masih kebingungan karena terjadi peloncatan versi dari versi 5 ke versi 7. PHP berkembang dari proyek experimen yang dinamakan PHPNG(PHP Next Generation). Proyek PHPNG bertujuan untuk menulis ulang kode PHP untuk meningkatkan perfoma. Dari proyek ini perfoma ini berhasil 100\% dari versi sebelumnya sehingga menamainya versi 7.
 \end{enumerate}
\subsection{Kelebihan Bahasa Pemrograman PHP}
  \begin{itemize}
    \item Mudah dan Serba Guna
      Seperti yang telah dijelaskan sebelumnya, kelebihan bahasa pemrograman PHP ini salah satunya adalah mudah untuk digunakan dan mendukung banyak kegunaan. Bahasa pemrograman ini dapat digunakan dengan mudah untuk membuat sisi server dari laman yang kita kembangkan dan penggunaan lainnya. Bahkan, bahasa pemrograman ini mendukung berbagai macam bahasa lain, seperti CSS dan JavaScript.
      Dan untuk mengembangkan produk dengan menggunakan bahasa pemrograman ini, telah terdapat banyak framework yang dapat digunakan. Salah satunya adalah Laravel yang menjadi framework PHP populer di kalangan developer web. Kelebihan ini tentu akan memberikan kemudahan bagi para developer dalam membangun produk-produk berbasis bahasa pemrograman PHP.
    \item Memiliki Komunitas yang Besar
      Salah satu kriteria bahasa pemrograman yang tepat untuk dipelajari adalah bahasa pemrograman tersebut harus memiliki komunitas yang besar. Salah satu alasannya adalah karena dengan komunitas yang besar, kita dapat belajar dengan mudah dengan adanya konunitas tersebut.
      Salah satu kelebihan bahasa pemrograman PHP menawarkan komunitas yang sangat besar. Tidak hanya besar, komunitas dari bahasa pemrograman PHP merupakan komunitas yang sangat aktif. Bahkan setiap kebanyakan masalah yang ada pada bahasa pemrograman ini telah memiliki solusi yang telah ada sebelumnya.
    \item Mendukung Database
     Bahasa pemrograman PHP memiliki kelebihan yang dapat digunakan untuk menangani database dengan sangat baik. Seperti yang telah dijelaskan sebelumnya, bahasa ini dapat digunakan untuk mengedit, menambahkan, atau bahkan digunakan untuk menghapus data pada database yang kita miliki.
     Bahasa pemrograman ini dapat bekerja dengan sangat baik untuk menangani berbagai hal yang berkaitan dengan file system, output HTML, images, pdfs, swf files, dan xhtml.
  \end{itemize}
\subsection{Penerapan PHP sebagai Bahasa Pemprograman}
Fungsi bahasa pemrograman php sendiri untuk web digunakan untuk dapat menyesuaikan tanpilan konten sesuai dengan situasi. Web yang bersifat dinamis juga digunakan untuk menyimpan data ke database dengan memproses from dan juga dapat megubah tampilan website sesuai inputan dari seorang user.PHP juga banyak diaplikasikan untuk pembuatan program-program seperti sistem informasi  klinik, rumah sakit, akademik, keuangan, manajemen aset, manajemen bengkel dan lain-lain. Dapat dikatakan bahwa program aplikasi yang dulunya hanya dapat dikerjakan untuk desktop aplikasi, PHP sudah dapat mengerjakannya.
Penerapan PHP saat ini juga banyak ditemukan pada proyek-proyek pemerintah seperti e-budgetting, e-procurement, e-goverment dan e e lainnya. Website Ubaya ini juga dibuat menggunakan PHP.
PHP juga dapat dilihat sebagai pilihan lain dari ASP.NET/C\#/VB.NET Microsoft, ColdFusion Macromedia, JSP/Java Sun Microsystems, dan CGI/Perl. Contoh aplikasi lain yang lebih kompleks berupa CMS yang dibangun menggunakan PHP adalah Wordpress, Mambo, Joomla, Postnuke, Xaraya, dan lain-lain.

\subsection{Sisi lain dari PHP  sebahgai Bahasa Pemprograman}
Menurut penulis yang sejak lama terlibat dalam pembuatan program dengan PHP ini adalah :
  \begin{itemize}
    \item Bahasa pemrograman PHP adalah sebuah bahasa script yang tidak perlu untuk dikompilasi (compile)
    \item Mudah diinstall ke dalam web server yang mendukung PHP seperti apache dengan konfigurasi yang mudah
    \item Dalam sisi pengembangan lebih mudah karena banyaknya milis-milis ataupun tutorial yang membahas tentang PHP
    \item PHP dapat dijalankan diberbagai sistem operasi, baik Windows, Linux, Macintosh.
  \end{itemize}
Meskipun bahasa pemrograman PHP tidak sepopuler bahasa pemrograman lainnya, bahasa pemrograman ini merupakan sebuah bahasa yang digunakan oleh perusahaan-perusahaan ternama, seperti Facebook dan IBM.
Kelebihan bahasa pemrograman PHP ini bahkan memikat perusahaan Facebook untuk membangun platform yang dimilikinya. Tidak hanya Facebook, perusahaan penyedia layanan web, yaitu WordPress, juga menggunakan bahasa pemrograman ini untuk mengembangkan platformnya.

Tidak hanya Facebook, perusahaan penyedia layanan web, yaitu WordPress, juga menggunakan bahasa pemrograman ini untuk mengembangkan platformnya.




