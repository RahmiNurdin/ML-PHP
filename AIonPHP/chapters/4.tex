
\subsection{Sejarah}
 Sejak pertama kali komputer diciptakan manusia sudah memikirkan bagaimana caranya agar komputer dapat belajar dari pengalaman. Hal tersebut terbukti pada tahun 1952, Arthur Samuel menciptakan sebuah program, game of checkers, pada sebuah komputer IBM. Program tersebut dapat mempelajari gerakan untuk memenangkan permainan checkers dan menyimpan gerakan tersebut kedalam memorinya.
\par Istilah machine learning pada dasarnya adalah proses komputer untuk belajar dari data (learn from data). Tanpa adanya data, komputer tidak akan bisa belajar apa-apa. Oleh karena itu jika kita ingin belajar machine learning, pasti akan terus berinteraksi dengan data. Semua pengetahuan machine learning pasti akan melibatkan data. Data bisa saja sama, akan tetapi algoritma dan pendekatan nya berbeda-beda untuk mendapatkan hasil yang optimal.

\subsection{Perbedaan Data Mining/Penggalian Data}
 Data mining adalah sebuah proses untuk menemukan pengetahuan, ketertarikan, dan pola baru dalam bentuk model yang deskriptif, dapat dimengerti, dan prediktif dari data dalam skala besar. Dengan kata lain data mining merupakan ekstraksi atau penggalian pengetahuan yang diinginkan dari data dalam jumlah yang sangat besar.
Dari definisi diatas dapat disimpulkan bahwa pada pembelajaran mesin berkaitan dengan studi, desain dan pengembangan dari suatu algoritma yang dapat memampukan sebuah komputer dapat belajar tanpa harus diprogram secara eksplisit. Sedangkan pada data mining dilakukan proses yang dimulai dari data yang tidak terstruktur lalu diekstrak agar mendapatkan suatu pengetahuan ataupun sebuah pola yang belum diketahui. Selama proses data mining itulah algoritma dari pembelajaran mesin digunakan.


\subsection Tipe Machine Learning Algoritma
 Machine Learning merupakan salah satu cabang dari disiplin ilmu Kecerdasan Buatan (Artificial Intellegence) yang membahas mengenai pembangunan sistem yang berdasarkan pada data. Banyak hal yang dipelajari, akan tetapi pada dasarnya ada 4 hal pokok yang dipelajari dalam machine learning.
\begin{enumerate}
 \item Pembelajaran Terarah (Supervised Learning)membuat fungsi yang memetakan masukan ke keluaran yang dikehendaki. Misalnya pengelompokan (klasifikasi). Merupakan algoritma yang belajar berdasarkan sekumpulan contoh pasangan masukan-keluaran yang diinginkan dalam jumlah yang cukup besar. Algoritma ini mengamati contoh-contoh tersebut dan kemudian menghasilkan sebuah model yang mampu memetakan masukan yang baru menjadi keluaran yang tepat.
       \par Salah satu contoh yang paling sederhana adalah terdapat sekumpulan contoh masukan berupa umur seseorang dan contoh keluaran yang berupa tinggi badan orang tersebut. Algoritma pembelajaran melalui contoh mengamati contoh-contoh tersebut dan kemudian mempelajari sebuah fungsi yang pada akhirnya dapat “memperkirakan” tinggi badan seseorang berdasarkan masukan umur orang tersebut.
 \item Pembelajaran Tak Terarah (Unsupervised Learning) memodelkan himpunan masukan, seperti penggolongan (clustering).Algoritma ini mempunyai tujuan untuk mempelajari dan mencari pola-pola menarik pada masukan yang diberikan. Meskipun tidak disediakan keluaran yang tepat secara eksplisit. Salah satu algoritma unsupervised learning yang paling umum digunakan adalah clustering/pengelompokan .
       \par Contoh unsupervised learning dalam dunia nyata misalnya  seorang supir taksi yang secara perlahan-lahan menciptakan konsep “macet” dan “tidak macet” tanpa pernah diberikan contoh oleh siapapun .
 \item Pembelajaran Semi Terarah (Semi-supervised Learning)ipe ini menggabungkan antara Supervised dan Unsupervised untuk menghasilkan suatu fungsi.
       \par Algoritma pembelajaran semi terarah menggabungkan kedua tipe algoritma di atas, di mana diberikan contoh masukan-keluaran yang tepat dalam jumlah sedikit dan sekumpulan masukan yang keluarannya belum diketahui. Algoritma ini harus membuat sebuah rangkaian kesatuan antara dua tipe algoritma di atas untuk dapat menutupi kelemahan pada masing-masing algoritma.
       \par Misalnya sebuah sistem yang dapat menebak umur seseorang berdasarkan foto orang tersebut.  Sistem tersebut membutuhkan beberapa contoh, misalnya yang didapatkan dengan mengambil foto seseorang dan menanyakan umurnya (pembelajaran terarah). Akan tetapi, pada kenyataannya beberapa orang sering kali berbohong tentang umur mereka sehingga menimbulkan noise pada data. Oleh karena itu, digunakan juga pembelajaran tak terarah agar dapat saling menutupi kelemahan masing-masing, yaitu noise pada data dan ketiadaan contoh masukan-keluaran.
 \item Reinforcement Learning Tipe ini mengajarkan bagaimana cara bertindak untuk menghadapi suatu masalah, yang suatu tindakan itu mempunyai dampak. 
       \par Adalah sebuah algoritma pembelajaran yang diterapkan pada agen cerdas agar ia dapat menyesuaikan dengan kondisi dilingkungannya, hal ini dicapai dengan cara memaksimalkan nilai dari hadiah ‘reward’ yang dapat dicapai. Suatu hadiah didefinisikan sebuah tanggapan balik ‘feedback’ dari tindakan agen bahwa sesuatu baik terjadi.Sebagai contoh, sangatlah sulit untuk memrogram sebuah agen untuk menerbangkan sebuah helikopter, tetapi dengan memberikan beberapa nilai negatif untuk menabrak, bergoyang-goyang, serta melenceng dari jalur tujuan perlahan-lahan agen tersebut dapat belajar menerbangkan helikopter dengan lebih baik.
\end{enumerate}

\subsection{Contoh Penerapan Machine Learning}
 Contoh penerapan machine learning dalam kehidupan adalah sebagai berikut :
\begin{enumerate}
 \item Penerapan di bidang kedoteran contohnya adalah mendeteksi penyakit seseorang dari gejala yang ada. Contoh lainnya adalah mendeteksi penyakit jantung dari rekaman elektrokardiogram.
 \item Pada bidang computer vision contohnya adalah penerapan pengenalan wajah dan pelabelan wajah seperti pada facebook. Contoh lainnya adalah penterjemahan tulisan tangan menjadi teks.
 \item Pada biang information retrival contohnya adalah penterjemahan bahasa dengan menggunakan komputer, mengubah suara menjadi teks, dan filter email spam.
\end{enumerate}

Salah satu teknik pengaplikasian machine learning adalah supervised learning. Seperti yang dibahas sebelumnya, machine learning tanpa data maka tidak akan bisa bekerja. Oleh karena itu hal yang pertama kali disiapkan adalah data. Data biasanya akan dibagi menjadi 2 kelompok, yaitu data training dan data testing. Data training nantinya akan digunakan untuk melatih algoritma untuk mencari model yang cocok, sementara data testing akan dipakai untuk mengetes dan mengetahui performa model yang didapatkan pada tahapan testing.
\par Dari model yang didapatkan, kita dapat melakukan prediksi yang dibedakan menjadi dua macam, tergantung tipe keluarannya. Jika hasil prediksi bersifat diskrit, maka dinamakan proses klasifikasi. Contohnya klasifikasi jenis kelamin dilihat dari tulisan tangan (output laki dan perempuan). Sementara jika kelurannya bersifat kontinyu, maka dinamakan proses regresi. Contohnya prediksi kisaran harga rumah di kota Bandung (output berupa harga rumah).

\subsection{Dampak Machine Learning di Masyarakat}
Penerapan teknologi machine learning mau tidak mau pasti telah dirasakan sekarang. Setidaknya ada dua dampak yang saling bertolak belakang dari pengembangan teknolgi machine learning. Ya, dampak positif dan dampak negatif.
\par Salah satu dampak positif dari machine learning adalah menjadi peluang bagi para wirausahawan dan praktisi teknologi untuk terus berkarya dalam mengembangkan teknologi machine learning. Terbantunya aktivitas yang harus dilakukan manusia pun menjadi salah satu dampak positif machine learning. Sebagai contohnya adalah adanya fitur pengecekan ejaan untuk tiap bahasa pada Microsoft Word. Pengecekan secara manual akan memakan waktu berhari-hari dan melibatkan banyak tenaga untuk mendapatkan penulisan yang sempurna. Tapi dengan bantuan fitur pengecekan ejaan tersebut, secara real-time kita bisa melihat kesalahan yang terjadi pada saat pengetikan.
\par Akan tetapi disamping itu ada dampak negatif yang harus kita waspadai. Adanya pemotongan tenaga kerja karena pekerjaan telah digantikan oleh alat teknologi machine learning adalah suatu permasalahan yang harus dihadapi. Ditambah dengan ketergantungan terhadap teknologi akan semakin terasa. Manusia akan lebih terlena oleh kemampuan gadget-nya sehingga lupa belajar untuk melakukan suatu aktivitas tanpa bantuan teknologi.

\section{ML on PHP}
\par Beberapa pendekatan yang dapat dimanfaatkan  untuk ML dalam PHP. Misalnya Algoritma, Validasi Silang, Jaringan Saraf Tiruan, Pra\-pemrosesan, Ekstraksi Fitur, dan banyak lagi contoh lainnya.

\subsection{Fitur ML on PHP}
Beberapa fitur yang dapat diterapkan dengan ML on PHP:
\begin{enumerate}
\item Association rule Learning
\begin{itemize}
\item Apriori
\end{itemize}
\item Classification
\begin{itemize}
\item SVC
\item k-Nearest Neighbors
\item Naive Bayes
\end{itemize}
\item Regression
\begin{itemize}
\item Least Squares
\item SVR
\end{itemize}
\item Clustering
\begin{itemize}
\item k-Means
\item DBSCAN
\end{itemize}
\item Metric
\begin{itemize}
\item Accuracy
\item Confusion Matrix
\item  Classification Report
\end{itemize}
\item Workflow
\begin{itemize}
\item  Pipeline
\end{itemize}
\item Neural Network
\begin{itemize}
\item Multilayer Perceptron Classifier
\end{itemize}
\item Cross Validation
\begin{itemize}
\item Random Split
\item Stratified Random Split
\end{itemize}
\item Feature Selection
\begin{itemize}
\item Variance Threshold
\item SelectKBest
\end{itemize}
\item Preprocessing
\begin{itemize}
\item Normalization
\item Imputation missing values
\end{itemize}
\item Feature Extraction
\begin{itemize}
\item Token Count Vectorizer
\item Tf-idf Transformer
\end{itemize}
\item Datasets
\begin{itemize}
\item Array
\item CSV
\item Files
\item SVM
\item MNIST
\end{itemize}
\item Models management
\begin{itemize}
\item Persistency
\end{itemize}
\item Math
\begin{itemize}
\item Distance
\item Matrix
\item Set
\item Statistic
\end{itemize}
\end{enumerate}


\subsection{Berkontribusi ke PHP\-ML}
\par PHP-ML adalah proyek sumber terbuka. Jika Anda ingin berkontribusi, silakan baca teks berikut ini. Sebelum saya dapat menggabungkan Permintaan Tarik Anda, berikut adalah beberapa panduan yang perlu Anda ikuti. Panduan ini ada untuk tidak mengganggu Anda, tetapi untuk menjaga basis kode tetap bersih, terpadu, dan bukti di masa mendatang.
\begin{enumerate}
\item Cabang
Anda hanya harus membuka permintaan tarik terhadap cabang master.
\item Tes Unit
Coba tambahkan tes untuk permintaan tarik Anda. Anda dapat menjalankan unit\-test dengan skript:
	vendor/bin/phpunit
\item Tes Kinerja
Sebelum menjalankan skrip bootstrap pertama kali, akan mengunduh semua set data yang diperlukan dari repositori publik : php\-ai/php\-m\l-datasets.
Tes kinerja waktu:
vendor/bin/phpbench run\-\-report=time
Tes kinerja memori:
vendor/bin/phpbench run \-\-report=memory
\item Travis
GitHub secara otomatis menjalankan permintaan tarik Anda melalui Travis CI. Jika Anda melanggar tes, saya tidak dapat menggabungkan kode Anda, jadi pastikan kode Anda berfungsi sebelum membuka Permintaan Tarik.
\item Menggabungkan
Tolong beri saya waktu untuk meninjau permintaan tarik Anda. Saya akan memberikan yang terbaik untuk meninjau semuanya secepat mungkin, tetapi tidak selalu sesuai dengan harapan saya.
\item Standar Pengkodean \& Analisis Statis
Saat berkontribusi kode ke PHP\-ML, Anda harus mengikuti standar pengkodeannya. Untuk melakukannya, jalankan:
	composer fix\-cs
\end{enumerate}

















