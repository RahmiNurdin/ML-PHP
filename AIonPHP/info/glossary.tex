\term{Hardware}Merupakan komponen dari sebuah komputer yang sifatnya bisa dapat dilihat secara kasat mata dan bisa diraba secara langsung dan hadware berfungsi untuk mendukung proses berjalanya komputer.

\term{Software}Merupakan suatu bagian dari sistem komputer yang tidak memiliki wujud fisik seperti hardware tetapi software merupakan sebuah nyawa komputer supaya dapat menjalankan perintah dari user.

\term{Internet}Merupakan jaringan komputer yang dimana satu jaringan dengan yang lain dapat saling terhubung untuk keperluan komunikasi dan informasi atau dapat disimpulkan internet dapat menghubungkan suatu media elektronik dengan media lainya.

\term{Server}Adalah sebuah sistem komputer yang menyediakan jenis layanan (service) tertentu dalam sebuah jaringan komputer server juga menjalankan perangkat lunak administratif yang mengontrol akses terhadap jaringan tersebut.

\term{client}Sistem atau proses yang dapat melakukan permintaan (request) data ke server.

\term{broadcast}Adalah sebuah pengiriman data dimana data akan dikirim langsung ke banyak titik sekaligus tanpa melakukan pengecekan,
Broadcast merupakan sebuah pengiriman data dimana data akan dikirim ke titik yang tidak sedikit secara bersamaan.

\term{switch}Sebuah perangkat jaringan pada komputer yang menghubungkan sebuah perangkat pada sebuah jaringan komputer dengan menggunakan pertukaran paket untuk menerima data, dan akan memproses untuk meneruskan data ke perangkat yang akan dituju.

\term{Hub}Adalah sebuah perangkat yang berfungsi untuk menghubungkan komputer yang satu dengan komputer lainnya dalam suatu sistem jaringan. 

\term{Bridge}Merupakan sebuah komponen jaringan yang banyak dipergunakan untuk memperluas jaringan atau membuat segmen jaringan.
