%%%%%%%%%%%%%%
%% Run LaTeX on this file several times to get Table of Contents,
%% cross-references, and citations.

%% If you have font problems, you may edit the w-bookps.sty file
%% to customize the font names to match those on your system.

%% w-bksamp.tex. Current Version: Feb 16, 2012
%%%%%%%%%%%%%%%%%%%%%%%%%%%%%%%%%%%%%%%%%%%%%%%%%%%%%%%%%%%%%%%%
%
%  Sample file for
%  Wiley Book Style, Design No.: SD 001B, 7x10
%  Wiley Book Style, Design No.: SD 004B, 6x9
%
%
%  Prepared by Amy Hendrickson, TeXnology Inc.
%  http://www.texnology.com
%%%%%%%%%%%%%%%%%%%%%%%%%%%%%%%%%%%%%%%%%%%%%%%%%%%%%%%%%%%%%%%%

%%%%%%%%%%%%%
% 7x10
%\documentclass{wileySev}

% 6x9
\documentclass{wileySix}

\usepackage{graphicx}
\usepackage{listings}

\usepackage{color}
 
\definecolor{codegreen}{rgb}{0,0.6,0}
\definecolor{codegray}{rgb}{0.5,0.5,0.5}
\definecolor{codepurple}{rgb}{0.58,0,0.82}
\definecolor{backcolour}{rgb}{0.95,0.95,0.92}
 
\lstdefinestyle{mystyle}{
    backgroundcolor=\color{backcolour},   
    commentstyle=\color{codegreen},
    keywordstyle=\color{magenta},
    numberstyle=\tiny\color{codegray},
    stringstyle=\color{codepurple},
    basicstyle=\footnotesize,
    breakatwhitespace=false,         
    breaklines=true,                 
    captionpos=b,                    
    keepspaces=true,                 
    numbers=left,                    
    numbersep=5pt,                  
    showspaces=false,                
    showstringspaces=false,
    showtabs=false,                  
    tabsize=2,
    language=sh
}
 
\lstset{style=mystyle}

%%%%%%%
%% for times math: However, this package disables bold math (!)
%% \mathbf{x} will still work, but you will not have bold math
%% in section heads or chapter titles. If you don't use math
%% in those environments, mathptmx might be a good choice.

% \usepackage{mathptmx}

% For PostScript text
\usepackage{w-bookps}

%%%%%%%%%%%%%%%%%%%%%%%%%%%%%%%%%%%%%%%%%%%%%%%%%%%%%%%%%%%%%%%%
%% Other packages you might want to use:

% for chapter bibliography made with BibTeX
% \usepackage{chapterbib}

% for multiple indices
% \usepackage{multind}

% for answers to problems
% \usepackage{answers}

%%%%%%%%%%%%%%%%%%%%%%%%%%%%%%
%% Change options here if you want:
%%
%% How many levels of section head would you like numbered?
%% 0= no section numbers, 1= section, 2= subsection, 3= subsubsection
%%==>>
\setcounter{secnumdepth}{3}

%% How many levels of section head would you like to appear in the
%% Table of Contents?
%% 0= chapter titles, 1= section titles, 2= subsection titles, 
%% 3= subsubsection titles.
%%==>>
\setcounter{tocdepth}{2}

%% Cropmarks? good for final page makeup
%% \docropmarks

%%%%%%%%%%%%%%%%%%%%%%%%%%%%%%
%
% DRAFT
%
% Uncomment to get double spacing between lines, current date and time
% printed at bottom of page.
% \draft
% (If you want to keep tables from becoming double spaced also uncomment
% this):
% \renewcommand{\arraystretch}{0.6}
%%%%%%%%%%%%%%%%%%%%%%%%%%%%%%

%%%%%%% Demo of section head containing sample macro:
%% To get a macro to expand correctly in a section head, with upper and
%% lower case math, put the definition and set the box 
%% before \begin{document}, so that when it appears in the 
%% table of contents it will also work:

\newcommand{\VT}[1]{\ensuremath{{V_{T#1}}}}

%% use a box to expand the macro before we put it into the section head:

\newbox\sectsavebox
\setbox\sectsavebox=\hbox{\boldmath\VT{xyz}}

%%%%%%%%%%%%%%%%% End Demo


\begin{document}


\booktitle{Machine Learning Dengan PHP }
%\booktitle{Pada Pemrograman PHP}
\subtitle{Teori dan Praktek}

\authors{Muhammad Yusril Helmi Setyawan\\
\affil{Informatics Research Center}
%Floyd J. Fowler, Jr.\\
%\affil{University of New Mexico}
}

\offprintinfo{Machine Learning Dengan PHP, Cetakan Pertama}{Muhammad Yusril Helmi Setyawan}

%% Can use \\ if title, and edition are too wide, ie,
%% \offprintinfo{Survey Methodology,\\ Second Edition}{Robert M. Groves}

%%%%%%%%%%%%%%%%%%%%%%%%%%%%%%
%% 
\halftitlepage

\titlepage


\begin{copyrightpage}{2019}
%Survey Methodology / Robert M. Groves . . . [et al.].
%\       p. cm.---(Wiley series in survey methodology)
%\    ``Wiley-Interscience."
%\    Includes bibliographical references and index.
%\    ISBN 0-471-48348-6 (pbk.)
%\    1. Surveys---Methodology.  2. Social 
%\  sciences---Research---Statistical methods.  I. Groves, Robert M.  II. %
%Series.\\
%
%HA31.2.S873 2007
%001.4'33---dc22                                             2004044064
\end{copyrightpage}

\dedication{`Jika Kamu tidak dapat menahan lelahnya belajar, 
Maka kamu harus sanggup menahan perihnya Kebodohan.'
~Imam Syafi'i~}

\begin{contributors}
\name{Rolly Maulana Awangga,} Informatics Research Center., Politeknik Pos Indonesia, Bandung,
Indonesia



\end{contributors}

\contentsinbrief
\tableofcontents
\listoffigures
\listoftables
\lstlistoflistings


\begin{foreword}
Sepatah kata dari Kaprodi, Kabag Kemahasiswaan dan Mahasiswa
\end{foreword}

\begin{preface}
Buku ini adalah pengantar pemahaman tentang implementasi bahasa pemrograman PHP pada Machine Learning

\prefaceauthor{Muhammad Yusril Helmi Setyawan}
\where{Bandung, Jawa Barat\\
Februari, 2019}
\end{preface}


\begin{acknowledgments}
Terima Kasih kepada seluruh dosen di program studi Diploma IV Teknik Informatika Politeknik Pos Indonesia yang memberikan kritik dan saran atas isi buku ini, juga para mahasiswa yang saya banggakan yang telah memberikan masukan-masukan agar buku ini menjadi mudah dipahami.
\authorinitials{Y. H. S.}
\end{acknowledgments}

\begin{acronyms}
\acro{ML}{Machine Learning}




\end{acronyms}

\begin{glossary}
\term{Hardware}Merupakan komponen dari sebuah komputer yang sifatnya bisa dapat dilihat secara kasat mata dan bisa diraba secara langsung dan hadware berfungsi untuk mendukung proses berjalanya komputer.

\term{Software}Merupakan suatu bagian dari sistem komputer yang tidak memiliki wujud fisik seperti hardware tetapi software merupakan sebuah nyawa komputer supaya dapat menjalankan perintah dari user.

\term{Internet}Merupakan jaringan komputer yang dimana satu jaringan dengan yang lain dapat saling terhubung untuk keperluan komunikasi dan informasi atau dapat disimpulkan internet dapat menghubungkan suatu media elektronik dengan media lainya.

\term{Server}Adalah sebuah sistem komputer yang menyediakan jenis layanan (service) tertentu dalam sebuah jaringan komputer server juga menjalankan perangkat lunak administratif yang mengontrol akses terhadap jaringan tersebut.

\term{client}Sistem atau proses yang dapat melakukan permintaan (request) data ke server.

\term{broadcast}Adalah sebuah pengiriman data dimana data akan dikirim langsung ke banyak titik sekaligus tanpa melakukan pengecekan,
Broadcast merupakan sebuah pengiriman data dimana data akan dikirim ke titik yang tidak sedikit secara bersamaan.

\term{switch}Sebuah perangkat jaringan pada komputer yang menghubungkan sebuah perangkat pada sebuah jaringan komputer dengan menggunakan pertukaran paket untuk menerima data, dan akan memproses untuk meneruskan data ke perangkat yang akan dituju.

\term{Hub}Adalah sebuah perangkat yang berfungsi untuk menghubungkan komputer yang satu dengan komputer lainnya dalam suatu sistem jaringan. 

\term{Bridge}Merupakan sebuah komponen jaringan yang banyak dipergunakan untuk memperluas jaringan atau membuat segmen jaringan.

\end{glossary}

\begin{symbols}
\term{A}Amplitude

\term{\hbox{\&}}Propositional logic symbol 

\term{a}Filter Coefficient

\bigskip

\term{\mathcal{B}}Number of Beats
\end{symbols}

\begin{introduction}

%% optional, but if you want to list author:

\introauthor{Rolly Maulana Awangga, S.T., M.T.}
{Informatics Research Center\\
Bandung, Jawa Barat, Indonesia}

Pada era disruptif  \index{disruptif}\index{disruptif!modern} 
saat ini. git merupakan sebuah kebutuhan dalam sebuah organisasi pengembangan perangkat lunak.
Buku ini diharapkan bisa menjadi penghantar para programmer, analis, IT Operation dan Project Manajer.
Dalam melakukan implementasi git pada diri dan organisasinya.

Rumusnya cuman sebagai contoh aja biar keren\cite{awangga2018sampeu}.

\begin{equation}
ABC {\cal DEF} \alpha\beta\Gamma\Delta\sum^{abc}_{def}
\end{equation}

\end{introduction}

%%%%%%%%%%%%%%%%%%Isi Buku_

\chapter{Pengantar Machine learning}
\section{Pengenalan Jaringan Komputer}
 Jaringan Komputer merupakan kumpulan dari beberapa PC(Personal Computer) atau peripheral yang saling terhubung melalui media transmisi(melalui kabel atau nirkabel) dan melakukan akses bersama  terhadap suatu resource.
 Secara lebih sederhana, jaringan komputer dapat diartikan sebagai sekumpulan komputer berserta mekanisme dan prosedurnya yang saling terhubung dan berkomunikasi.  Komunikasi yang dilakukan oleh komputer tersebut dapat berupa transfer berbagai data, instruksi, dan informasi dari satu komputer ke komputer yang lain \cite{irawan2012analisis}.

 resource(sumber daya) tersebut terdiri dari:
 \begin{enumerate}
   \item Hardware, seperti: Printer,mesin fax, store device.
   \item Software, seperti: game, pemprograman client server, multi user, mail server
   \item Stored, Seperti: frontend atau backend
   \item Internet,Seperti: dial atau wireless
 \end{enumerate}

Keuntungan Jaringan Komputer:
\begin{enumerate}
  \item Lebih hemat dalam biaya pengadaan dan pemeliharaan
  \item Memungkinkan management sumber daya lebih efisien
  \item Mempertahankan kualitas Informasi agar tatap handal
  \item Memungkinkan Kelompok kerja berkomunikasi lebih efisien
  \item Keamanan data lebih terjamin
\end{enumerate}

Type Jaringan Komputer
 Pada dasarnya seseorang menentukan type jaringan komputer karena beberapa alasan, diantaranya adalah:
 \begin{enumerate}
   \item Disesuaikan dengan kebutuhan kita dalam membuat jaringan komputer.
   \item Tergantung kepada jumlah pengguna yang akan melakukan sharing data.
   \item keamanan (securitas) dari masing-masing jaringan.
   \item Mempertimbangkan dalam biaya pengadaan dari jaringan komputer
   \item Sumber daya admin menentukan jaringan komputer.
   \item Bentuk dari organisasi yang terbentuk.
 \end{enumerate}

Server Based
pada type jaringan komputer server based di perlukan satu atau lebih komputer khusus yang di sebut server untuk mengatur lalu lintas data atau informasi dalam jaringan komputer.komputer-komputer selain server dinamakan client. server yaitu komputer yang menyediakan fasilitas bagi komputer-komputer lain, sedangkan client yaitu komputer-komputer yang menerima atau menggunakan fasilitas yang di sediakan oleh server \cite{yudianto2007jaringan}.

server dibedakan atas dua macam yaitu dedicated server(server bisa jadi client) dan undedicated server(server mutlak,tidak bisa jadi client).

macam-macam undedicated server:
\begin{itemize}
  \item DNS (Domain Name Service) yaitu server yang di gunakan untuk mengkonfersi penamaan IP address menjadi penanaman yang lebih familier (umum).
  \item DHCP (Dinamic Host Configurasi Protocol) yaitu server yang di gunakan untuk memberikan pengalaman IP address secara otomatis yang bersifat random. cara kerja random. Cara kerjanya pertama request (permintaan) kemudian dibroadcast.
  \item  FTP (file Transfer Protokol) yaitu server yang di gunakan untuk mengola jenis file/folder supaya data yang dinformasikan terpusat
  \item Mail Server merupakan jenis data dalam bentuk surat elektronik dibedakan menjadi dua yitu dalam bentuk text POP V3 (post office protocol) dan dalam bentuk web SMTP (simple Mail Transfer Protocol)
  \item Web server yaitu server yang di gunakan untuk mengelola data web yang bersifat dinamis.
  \item Database server yaitu server dalam bentuk file database.
\end{itemize}

ciri-ciri Server based
\begin{itemize}
  \item Operating System yang di gunakan berjenis network
  \item Perangkat yang di gunakan lebih dari 10 PC
  \item Terdapat komputer yang di jadikan sebagai pengontrol(server)\cite{wahyono2007building}
\end{itemize}

kelebihan Server based
\begin{itemize}
  \item terpusatnya penyedia resource
  \item Sharing data lebih efektif dan efesien
  \item System keamanan dan admistrasi jaringan lebih baik
\end{itemize}

Ciri-ciri Server Based
\begin{itemize}
  \item Operating System yang di gunakan berjenis network
  \item Perangkat yang di gunakan lebih dari 10 PC
  \item Terdapat Komputer yang di jadikan sebagai pengontrol(server)
\end{itemize}

Kelebihan Server Based
\begin{itemize}
  \item Terpusatnya penyedia resource
  \item Sharing data lebih efektif dan efesien
  \item System keamanan dan administrasi jaringan lebih baik
\end{itemize}

\subsection {Jenis-jenis Jaringan Komputer}
 Jenis jenis jaringan komputer dilihat berdasarkan ruang lingkup dan luas jangkuannya,di bedakan menjadi beberapa macam,yaitu:
\begin{itemize}
  \item Local Area Network(LAN)
   LAN adalah suatu system jaringan di mana setiao komputer atau perangkat keras dan perangkat lunak di gabungkan agar dapat saling berkomunikasi (terintegrasi) dalam area kerja tertentu dengan menggunakan data dan program yang sama,juga mempunyai kecepatan transfer data lebih cepat. Ruang Lingkup LAN anatr ruangan,gedung,kantor
\end{itemize}

\subsection {Topologi Jaringan Komputer}
    Topologi jaringan komputer adalah jaringan yang berhubungan dengan susuanan fisik semua jaringan komputer, baik server maupun client yang menentukan design,susunan,bentuk dari cara penempatan komputer(peripheral) kedalam jaringan-jaringan komputer.Topologi akan membentuk:
\begin{enumerate}
  \item Jenis alat yang di gunakan
  \item kemampuan dari peralatan
  \item Pertembuhan dari jaringan komputer
  \item Bagaimana jaringan tersebut diatur
\end{enumerate}

jenis alat-alat yang di gunakan,syaratnya:
\begin{itemize}
  \item Minimal 2 PC
  \item Adanya Operating System
  \item Adanya Network Interface Card(NIC)
  \item Driver NIC
  \item Media Transmisi
  \item Konsetrator(penghubung
  \item Access Point(tanpa kabel)
  \item Hub
  \item switch
  \item Repeater(Penguat signal)
  \item Router (Pembeda IP Address)
  \item Gatway(Perbedaan Arsitektur)
  \item Bridge (penghubung perbedaan topologi)
  \item Modern (modulasi de Modelator)
\end{itemize}

Topologi jaringan dibagi menjadi dua macam yaitu:
\begin{enumerate}
  \item Fisika
        Topologi ini menjelaskan tentang bentuk dari jaringan komputer yang dapat dilihat secara fisik/nyara.
\end{enumerate}

\begin{enumerate}
  \item Topologi Bus
        Masing-masing server dan workstastion di hubungkan pada sebuah kabel yang di sebut trunk atau backbone, kabel untuk menghubungkan jaringan ini biasanya menggunakan kabel Coaxial (kabel BNC). setiap server dan workstation yang di sambungkan pada bus menggunakan konektor T.
\end{enumerate}
 pada kedua ujung dari kabel harus diberi terminator berupa resistor yang memiliki resistansi khusus sebesar 50 Ohm yang berwujud sebuah konektor.Apabila resistansi kabel dibawah maupun di atas 50 Ohm,maka server tidak akan bisa bekerja secara maksimal dalam melayani jaringan,sehingga akses user dan client menjadi menurun.
 kelebihan jaringan topologi bus:
 \begin{itemize}
   \item Penggunaan kabel yang sedikit sehingga terlihat sederhana.
   \item Pengembangan jaringan mudah.
 \end{itemize}
 Kekurangan jaringan topologi
 \begin{itemize}
   \item Membutuhkan repeater untuk jarak jaringan yang terlalu jauh.
   \item jaringan akan terganggu apabila salah satu komputer mengalami kerusakan.
   \item Deteksi kesalahan sangat kecil sehingga apabila terjadi gangguan maka sulit sekali mencari kesalahan tersebut.
   \item Terjadi antrian data
 \end{itemize}

\begin{enumerate}
  \item Topologi Star
        Pada topologi in, setiap komputer(node) dalam jaringan terhubung ke sebuah pusat jaringan,yang biasa berupa hub,switch, dan juga berupa komputer. setiap workstasion dihubungkan ke server menggunakan suatu konsentrator. masing-masing workstastion tidak saling berhubungan.jadi setiap user yang terhubung ke server tidak akan dapat berinteraksi dan melakukan apa-apa sebelum server kita di hidupkan.apabila komputer server mati maka semua koneksi jaringan akan terputus.
\end{enumerate}
kelebihan jaringan topologi star:
\begin{itemize}
  \item Mudah dalam medeteksi kesalahan jaringan karena control jaringan terpusat.
  \item fleksibe dalam hal pemasangan jaringan baru tanpa mempengaruhi jaringan yang lain.
  \item apabila salah satu kabel koneksi user terpusat maka hanya user yang bersangkutan saja yang tidak akan berfungsi dan tidak mempengaruhi user yang lain.
\end{itemize}
kekurangan jaringan topologi star:
\begin{itemize}
  \item Boros dalam pemakaian kabel jika kita hubungkan dengan jaringan yang lebih besar dan luas.
  \item Control hanya terpusat pada hub/switch sehingga operasionalnya perlu ditangani secara khusus.
\end{itemize}

\begin{enumerate}
  \item topologi cincin atau yang sering disebut dengan ring topologi adalah topologi jaringan di mana setiap komputer yang terhubung membuat lingkaran. dengan artian setiap komputer yang terhubung kedalam satu jaringan saling terkoneksi ke dua komputer lainnya sehingga membentuk satu jaringan yang sama dengan bentuk cincin.
      Pada setiap komputer akan dihubungkan dan di jadikan repeater(penguat signal). komputer yang diberi frame berhak mengirim data dan komputer yang lain menjadi repeater. pada topologi ring terdapat token frame yang saling berkesinambungan dan pada prinsipnya menggunakan CSMA/CD(Carrier Sense Multyple Access/Collection Detection).
\end{enumerate}
Kelebihan jaringan topologi ring adalah:
\begin{itemize}
  \item Hemat kabel
  \item Dapat mengisolasi kesalahan dari suatu workstation kekurangan jaringan topolgi ring
  \item Sangat peka terhadap kesalahan jaringan walaupun sekecil apapun.
  \item Sukar untuk mengembangkan jaringan,sehingga jaringan tersebut tampak menjadi kaku.
  \item Biaya pemasangan Lebih besar.
\end{itemize}

\begin{enumerate}
  \item Logik
        sedanhkan topologi ini menjelaskan tentang bagaimana signal akan melewati komputer didalam jaringan. Arsitektur ini terus di kembangkan sampai saat ini.
\end{enumerate}

\begin{enumerate}
  \item  Token Ring
         Token ring memanfaatkan topologi ring. sebuah token bebas mengalir dalam jaringan. Apabila suatu node ingin mengirim paket data, maka paket data yang akan dikirim ditempel pada token, token kemudian akan membawa paket data tersebut pada tujuannya. pada waktu token terisi data, node lain tidak dapat menggunakan token tersebut sampai token menyelesaikan tugas mengirimkan paket data. apabila paket data telah di sampaikan pada tujuan,node pengguna tadi melepaskan token untuk dipakai oleh node lain. cara kerja dinamakan token passing scheme.
\end{enumerate}
ciri-ciri token ring:
\begin{itemize}
  \item Kecepatannya 1 Mbps, 4 Mbps hingga 16 Mbps.
  \item untuk menghubungkan station membutuhkan multistation Access Unit(MAU)
\end{itemize}

\begin{enumerate}
  \item Arsitektur ArcNet(Attached ResourceComputer Network)
        Didesain untuk system komputer Datapoint dan dikembangkan oleh Datapoint Corporation. Saat pertama didesain Arcnet menggunakan ukuran frame kecil 508 byte. Arcnet didesain agar handal dan tahan terhadap kerusakan pada kabel dan station.
\end{enumerate}
Ciri-ciri ArcNet adalah:
\begin{itemize}
  \item Topologi fisik yang di gunakan biasanya topologi Bus atau Star
  \item Prinsip kerjanya menggunakan token passing scheme atau broadcast
  \item Implementasinya menggunakan kabel coaxial RG-62
  \item Kecepatan mulai dari 2.5 Mbps hingga 20 Mbps.
\end{itemize}

\begin{enumerate}
  \item Merupakan implementasi metode CSMA/CD yang dikembangkan
        tahun 1960 pada proyek wire.sejak tahun 1978 IEEE (Institute Of Electrical and Electronics Engineers) telah melakukan
        standarisasi system Ethernet. Kecepatan Transmisi data saat ini antara 10 sampai 100 Mbps.
\end{enumerate}

\begin{enumerate}
  \item FDDI(Fiber Distribusi Data Interface)
        Merupakan suatu protocol jaringan yang menghubungkan antara
        dua atau beberapa jaringan yang jaraknya berdekatan ataupun berjauhan adapun metode yang di gunakan dalam FDDI adalah metode token ring.
\end{enumerate}

ciri-ciri FDDI adalah:
\begin{itemize}
  \item  Implementasinya menggunakan kabel fiber optic
  \item  Memiliki kecepatan 100 Mbps
  \item  Tidak compatibel dengan Ethernet tapi Ethernet dapat
         dienkapsulasi dalam paket FDDI
  \item  Bekerja berdasarkan dua ring concentris
  \item  Apabila salah satu ring atau node putus maka ring yang lain
         dapat berfungsi sebagai back up.
\end{itemize}

\begin{enumerate}
  \item ATM(Asynchronous Transfer MOde)
        Merupakan teknologi jaringan berkecepatan tinggi yang mampu
        mengirim data,suara dan video secara real time.ATM juga biasa di sebut Cell Relay. ATM merupakan interface transfer paket yang effesien. ATM menggunakan paket-paket dengan ukuran tertentu yang di sebut dengan cell. karena menggunakan ukuran tertentu ini, ATM menghasilkan skema yang efisien bagi pentransmisian pada jaringan berkecepatan tinggi. ATM menyediakan layanan real time dan non real time.
\end{enumerate}

\subsection{Topologi jaringan komputer}
Topologi merupakan suatu pola hubungan antara terminal dalam jaringan komputer.pola ini sangat erat kaitannya dengan metode access dan media pengiriman yang di gunakan. Topologi yang ada sangatlah tergantung dengan letak geografis dari masing-masing terminal,kualitas kontrol yang di butuhkan dalam komunikasi ataupun penyampaian pesan, serta kecepatan dari pengiriman data. dalam definisi topologi terbagi menjadi dua, yaitu Topologi logik (logical topology) yang menunjukan bagaimana suatu media di akses oleh host.

Adapun topologi fisik yang umum di gunakan dalam membangun sebuah jaringan adalah:
\begin{itemize}
  \item Point to point(Titik ke Titik)
        jaringan kerja titik ke titik merupakan jaringan kerja yang paling sederhana tetapi dapat di gunakan secara luas. begitu sederhananya jaringan ini, sehingga sering kali tidak dianggap sebagai suatu jaringan tetapi hanya merupakan komunikasi biasa.

        dalam hal ini, kedua simpul mempunyai kedudukan yang setingkat, sehingga simpul manapun dapat memulai dan mengendali hubungan dalam jaringan tersebut. data kirim dari satu simpul langsung kesimpul lainnya sebagai penerima,misalnya antara terminal dengan cpu.
  \item  star Network (jaringan bintang)
         dalam konfigurasi bintang, beberapa peralatan yang ada akan di hubungkan kedalam satu pusat komputer.kontrol yang ada akan di pusatkan pada sutu titik, seperti misalnya mengatur beban kerja serta pengaturan sumber daya yang ada. semua link harus berhubungan dengan pusat apabila ingin menyalurkan dara kesimpul lainnya yang di tuju. dalam hal ini, bila pusat mengalami gangguan, maka semua terminal juga akan terganggu. model jaringan bintang ini relative sangat sederhana, sehingga banyak di gunakan oleh pihak per-bank-kan yang biasanya mempunyai banyak kantor cabang yang tersebar di berbagai lokasi. dengan adanya konfigurasi bintang ini, maka segala macam kegiatan yang ada di kantor cabang dapatlah di kontrol dan di koordinasikan dengan baik. di samping itu, dunia pendidikan juga banyak memanfaatkan jaringan bintang ini guna mengontrol kegiatan anak didik mereka.
  \item Ring Networks (jaringan Cincin)
  \item    pada jaringan ini terdapat beberapa peralatan saling
        di hubungkan satu dengan lainnya dan pada akhirnya akan membentuk bagan seperti halnya sebuah cincin. jaringan cincin tidak memiliki suatu titik yang bertindak sebagai pusat ataupun pengantur lalu lintas data, semua simpul mempunyai tingkatan yang sama. data yang di kirim akan berjalan melewati beberapa simpul sehingga sampai pada simpul yang di tuju. dalam menyampaikan data, jaringan bisa bergerak dalam satu ataupun dua arah.

        walaupun demikian, data yang ada tetap bergerak satu arah dalam satu saat. pertama, pesan ada akan di sampaikan dari titik ke titik lainnya dalam satu arah. apabila di temui kegagalan, misalnya terdapat kerusakan pada peralatan yang ada, maka data yang akan di kirim dengan cara kedua, yaitu pesan kemudian di transmisi dalam arah yang berlawanan, dan pada akhirnya bisa berakhir pada tempat yang di tuju.
  \item tree Network (jaringan pohon)
        pada jaringan pohon, terdapat beberapa tingkatan simpul (node).pusat atau simpul yang lebih tinggi tingkatanya, dapat mengatur simpul lain yang lebih rendah tingkatanya.
\end{itemize} 


\chapter{Teknologi dengan Machine Learning}
\section{Sejarah Machine Learning}
Sejak pertama kali komputer diciptakan manusia sudah memikirkan bagaimana caranya agar komputer dapat belajar dari pengalaman. Hal tersebut terbukti pada tahun 1952, Arthur Samuel menciptakan 
sebuah program, game of checkers, pada sebuah komputer IBM. Program tersebut dapat mempelajari gerakan untuk memenangkan permainan checkers dan menyimpan gerakan tersebut kedalam memorinya.
Istilah machine learning pada dasarnya adalah proses komputer untuk belajar dari data (learn from data). Tanpa adanya data, komputer tidak akan bisa belajar apa-apa. Oleh karena itu jika kita ingin belajar machine learning, pasti akan terus berinteraksi dengan data. Semua pengetahuan machine learning pasti akan melibatkan data. Data bisa saja sama, akan tetapi algoritma dan pendekatan nya berbeda-beda untuk mendapatkan hasil yang optimal.
\begin{enumerate}
	\item pembelajaran terarah (Supervised Learning)
	\item pembelajaran tak terarah (Unsupervised Learning)
	\item Pembelajaran semi terarah (Semi-supervised Learning)
	\item Reinforcement Learning
\end{enumerate}


\chapter{PHP sebagai Bahasa Pemrograman}
\section{PHP sebagai Bahasa Pemprograman}
\subsection{Mengenal PHP sebagai Bahasa Pemprograman}
PHP adalah singkatan dari "PHP: Hypertext Prepocessor", yaitu bahasa pemrograman yang digunakan secara luas untuk penanganan pembuatan dan pengembangan sebuah situs web dan bisa digunakan bersamaan dengan HTML. PHP diciptakan oleh Rasmus Lerdorf pertama kali tahun 1994. Sejak versi 3.0, nama bahasa ini diubah menjadi "PHP: Hypertext Prepocessor" dengan singkatannya "PHP". PHP versi terbaru adalah versi ke-5.
Bahasa pemrograman ini termasuk ke dalam bahasa pemrograman yang serba guna dan mendukung terhadap PHP Code, Text, HTML, CSS, dan JavaScript. Bahasa pemrograman PHP juga mampu menangani banyak hal dalam pengembangan web.
Bahasa ini mampu mengumpulkan data serta membuat konten laman web menjadi lebih dinamis. Bahasa ini dapat digunakan untuk membuat, membuka, membaca, menulis, dan menutup file yang berada di sisi server. Bahasa pemrograman ini juga dapat menangani database, seperti menghapus, menambah, atau memodifikasi data.
Tidak hanya itu, bahasa ini juga dapat menangani keamanan dari data. Bahasa pemrograman ini dapat digunakan untuk membatasi pengguna untuk mengakses beberapa laman pada website yang dikembangkan. Bahasa pemrograman ini juga mampu mengenkripsi data yang ada.
\subsection{Sejarah PHP}
Pada awalnya PHP dikenal dengan singkatan Personal Home Page. Karena server tersebut di peruntukan untuk website pribadi. Tetapi untuk saat ini PHP sudah bermetamorfosis menjadi bahasa pemrograman yang sangat populer yang digunakan untuk website terkenal seperti Wikipedia,wordpress,joomla,dll.
Untuk saat ini php dikenal dengan singkatan Hypertext Preprocessor sebuah kepanjangan rekursif, yakni permainan kata dimana kepanjangannya terdiri dari singkatan itu sendiri. Bahasa pemrograman php banyak digunakan karena sifatnya yang open source yaitu dapat digunakan secara gratis.Bahasa Pemrograman PHP adalah bahasa pemrograman script server-side yang didesain untuk pengembangan web. Selain itu, PHP juga bisa digunakan sebagai bahasa pemrograman umum. Pertama kali di kembangkan oleh Rasmust Lerdorf pada tahun 1995, dan sekarang php dikembangkan oleh The PHP Group.
Sementara untuk penyisipan kode php dapat disisipkan pada html. Karena php bersifat Scripting Language atau Bahasa Pemprograman script. PHP sendiri memiliki perkembangan versi dari tahun ketahun di antaranya :
  \begin{enumerate}
     \item PHP/ FI : Personal Home Page / Forms Interfreter.
      Berasal dari tahun 1994 yang dikembangkan oleh Rasmus Lerdoft untuk membuat kode program (script) dengan Bahasa perl untuk web pribadinya. Salah satu kegunaan script ini adalah untuk menampilkan resume pribadi dan mencatat jumlah pengunjung ke sejumlah website.
     \item PHP/ FI : Personal Home Page / Form Interpreter 2
      Pada 1996 Rasmus Lerdoft mengumumkan PHP/FI versi 2.0. versi 2 ini dirancang lerdoft pada saat mengerjakan sebuah proyek di University of Toronto yang membutuhkan pengolahan data dan tampilan web yang rumit.
     \item PHP : Hypertext Prepocessor 3
      Terjadi pada pertengahan tahun 1997, telah banyak menarik perhatian programmer namun Bahasa ini memiliki masalah dengan kestabilan yang kurang bisa diandalkan.
     \item PHP : Hypertext Preprocessor 4
      Dalam fitur ini PHP memperkenalkan beberapa fitur lanjutan, seperti layer abstraksi antara PHP dan web server, menambahkan mekanisme thread-safety, dan two-stage parsing.
    \item PHP : Hypertext Preprocessor 5
      Versi PHP terakhir hingga saat ini, yaitu PHP 5.X diluncurkan pada 13 juli 2004. PHP 5 telah mendukung penuh pemrograman object dan peningkatan perfoma melalui Zend engine versi 2.
    \item PHP Hypertext Preprocessor 7
      Pada versi ini programmer masih kebingungan karena terjadi peloncatan versi dari versi 5 ke versi 7. PHP berkembang dari proyek experimen yang dinamakan PHPNG(PHP Next Generation). Proyek PHPNG bertujuan untuk menulis ulang kode PHP untuk meningkatkan perfoma. Dari proyek ini perfoma ini berhasil 100% dari versi sebelumnya sehingga menamainya versi 7.
 \end{enumerate}
\subsection{Kelebihan Bahasa Pemrograman PHP}
  \begin{itemize}
    \item Mudah dan Serba Guna
      Seperti yang telah dijelaskan sebelumnya, kelebihan bahasa pemrograman PHP ini salah satunya adalah mudah untuk digunakan dan mendukung banyak kegunaan. Bahasa pemrograman ini dapat digunakan dengan mudah untuk membuat sisi server dari laman yang kita kembangkan dan penggunaan lainnya. Bahkan, bahasa pemrograman ini mendukung berbagai macam bahasa lain, seperti CSS dan JavaScript.
      Dan untuk mengembangkan produk dengan menggunakan bahasa pemrograman ini, telah terdapat banyak framework yang dapat digunakan. Salah satunya adalah Laravel yang menjadi framework PHP populer di kalangan developer web. Kelebihan ini tentu akan memberikan kemudahan bagi para developer dalam membangun produk-produk berbasis bahasa pemrograman PHP.
    \item Memiliki Komunitas yang Besar
      Salah satu kriteria bahasa pemrograman yang tepat untuk dipelajari adalah bahasa pemrograman tersebut harus memiliki komunitas yang besar. Salah satu alasannya adalah karena dengan komunitas yang besar, kita dapat belajar dengan mudah dengan adanya konunitas tersebut.
      Salah satu kelebihan bahasa pemrograman PHP menawarkan komunitas yang sangat besar. Tidak hanya besar, komunitas dari bahasa pemrograman PHP merupakan komunitas yang sangat aktif. Bahkan setiap kebanyakan masalah yang ada pada bahasa pemrograman ini telah memiliki solusi yang telah ada sebelumnya.
    \item Mendukung Database
     Bahasa pemrograman PHP memiliki kelebihan yang dapat digunakan untuk menangani database dengan sangat baik. Seperti yang telah dijelaskan sebelumnya, bahasa ini dapat digunakan untuk mengedit, menambahkan, atau bahkan digunakan untuk menghapus data pada database yang kita miliki.
     Bahasa pemrograman ini dapat bekerja dengan sangat baik untuk menangani berbagai hal yang berkaitan dengan file system, output HTML, images, pdfs, swf files, dan xhtml.
  \end{itemize}
\subsection{Penerapan PHP sebagai Bahasa Pemprograman}
Fungsi bahasa pemrograman php sendiri untuk web digunakan untuk dapat menyesuaikan tanpilan konten sesuai dengan situasi. Web yang bersifat dinamis juga digunakan untuk menyimpan data ke database dengan memproses from dan juga dapat megubah tampilan website sesuai inputan dari seorang user.PHP juga banyak diaplikasikan untuk pembuatan program-program seperti sistem informasi  klinik, rumah sakit, akademik, keuangan, manajemen aset, manajemen bengkel dan lain-lain. Dapat dikatakan bahwa program aplikasi yang dulunya hanya dapat dikerjakan untuk desktop aplikasi, PHP sudah dapat mengerjakannya.
Penerapan PHP saat ini juga banyak ditemukan pada proyek-proyek pemerintah seperti e-budgetting, e-procurement, e-goverment dan e e lainnya. Website Ubaya ini juga dibuat menggunakan PHP.
PHP juga dapat dilihat sebagai pilihan lain dari ASP.NET/C#/VB.NET Microsoft, ColdFusion Macromedia, JSP/Java Sun Microsystems, dan CGI/Perl. Contoh aplikasi lain yang lebih kompleks berupa CMS yang dibangun menggunakan PHP adalah Wordpress, Mambo, Joomla, Postnuke, Xaraya, dan lain-lain.

\subsection{Sisi lain dari PHP  sebahgai Bahasa Pemprograman}
Menurut penulis yang sejak lama terlibat dalam pembuatan program dengan PHP ini adalah :
  \begin{itemize}
    \item Bahasa pemrograman PHP adalah sebuah bahasa script yang tidak perlu untuk dikompilasi (compile)
    \item Mudah diinstall ke dalam web server yang mendukung PHP seperti apache dengan konfigurasi yang mudah
    \item Dalam sisi pengembangan lebih mudah karena banyaknya milis-milis ataupun tutorial yang membahas tentang PHP
    \item PHP dapat dijalankan diberbagai sistem operasi, baik Windows, Linux, Macintosh.
  \end{itemize}
Meskipun bahasa pemrograman PHP tidak sepopuler bahasa pemrograman lainnya, bahasa pemrograman ini merupakan sebuah bahasa yang digunakan oleh perusahaan-perusahaan ternama, seperti Facebook dan IBM.
Kelebihan bahasa pemrograman PHP ini bahkan memikat perusahaan Facebook untuk membangun platform yang dimilikinya. Tidak hanya Facebook, perusahaan penyedia layanan web, yaitu WordPress, juga menggunakan bahasa pemrograman ini untuk mengembangkan platformnya.

Meskipun bahasa pemrograman PHP tidak sepopuler bahasa pemrograman lainnya, bahasa pemrograman ini merupakan sebuah bahasa yang digunakan oleh perusahaan-perusahaan ternama, seperti Facebook dan IBM.
Kelebihan bahasa pemrograman PHP ini bahkan memikat perusahaan Facebook untuk membangun platform yang dimilikinya. Tidak hanya Facebook, perusahaan penyedia layanan web, yaitu WordPress, juga menggunakan bahasa pemrograman ini untuk mengembangkan platformnya.




\chapter{Implementasi PHP pada Machine Learning}

\section{PHP-ML - Machine Learning library for PHP}
Pendekatan segar di php. machine learning Algoritma, , validasi silang , jaringan syaraf , pemroses fitur ekstraksi dan lebih banyak lagi Sebagai salah satu perpustakaan.Php-ml membutuhkan php \& gt\; = 7.1.fresh pendekatan untuk machine learning di php. Algoritma, validasi silang , jaringan syaraf , pemroses fitur ekstraksi dan lebih banyak lagi sebagai salah satu perpustakaan.
Php-ml membutuhkan php \& = 7.1.
Simple example of classification dapat kita lihat pada listing \ref{lst:code1}
\lstinputlisting[caption=contoh perintah 1,label={lst:code1}]{src/code1.tex}
\section{installation} 
Saat ini perpustakaan ini dalam proses pengembangan, tapi anda bisa memasangnya dengan komposer:

\begin{verbatim}composer require php-ai/php-ml\end{verbatim}

\section{examples}
Contoh script tersedia di repository yang terpisah php-ai/php-ml-examples.


\section{Features}
Ada beberapa fitur diantaranya :
\begin{enumerate}
\item Association rule Lerning
\item Classification
\item Regression
\item Clustering
\item Metric
\item Workflow
\item Neural Network
\item Cross Validation
\item Feature Selection
\item Preprocessing
\item Feature Extraction
\item Datasets
\item Models management
\item Math
\end{enumerate}

\section{Contribute}
\begin{enumerate}
\item Guide: CONTRIBUTING.md
\item Issue Tracker: github.com/php-ai/php-ml
\item Source Code: github.com/php-ai/php-ml
\end{enumerate}

\section{Machine Learning}

\subsection{Perbedaan Data Mining/Penggalian Data}
 Data mining adalah sebuah proses untuk menemukan pengetahuan, ketertarikan, dan pola baru dalam bentuk model yang deskriptif, dapat dimengerti, dan prediktif dari data dalam skala besar. Dengan kata lain data mining merupakan ekstraksi atau penggalian pengetahuan yang diinginkan dari data dalam jumlah yang sangat besar.
Dari definisi diatas dapat disimpulkan bahwa pada pembelajaran mesin berkaitan dengan studi, desain dan pengembangan dari suatu algoritma yang dapat memampukan sebuah komputer dapat belajar tanpa harus diprogram secara eksplisit. Sedangkan pada data mining dilakukan proses yang dimulai dari data yang tidak terstruktur lalu diekstrak agar mendapatkan suatu pengetahuan ataupun sebuah pola yang belum diketahui. Selama proses data mining itulah algoritma dari pembelajaran mesin digunakan.


\subsection Tipe Machine Learning Algoritma
 Machine Learning merupakan salah satu cabang dari disiplin ilmu Kecerdasan Buatan (Artificial Intellegence) yang membahas mengenai pembangunan sistem yang berdasarkan pada data. Banyak hal yang dipelajari, akan tetapi pada dasarnya ada 4 hal pokok yang dipelajari dalam machine learning.
\begin{enumerate}
 \item Pembelajaran Terarah (Supervised Learning)membuat fungsi yang memetakan masukan ke keluaran yang dikehendaki. Misalnya pengelompokan (klasifikasi). Merupakan algoritma yang belajar berdasarkan sekumpulan contoh pasangan masukan-keluaran yang diinginkan dalam jumlah yang cukup besar. Algoritma ini mengamati contoh-contoh tersebut dan kemudian menghasilkan sebuah model yang mampu memetakan masukan yang baru menjadi keluaran yang tepat.
       \par Salah satu contoh yang paling sederhana adalah terdapat sekumpulan contoh masukan berupa umur seseorang dan contoh keluaran yang berupa tinggi badan orang tersebut. Algoritma pembelajaran melalui contoh mengamati contoh-contoh tersebut dan kemudian mempelajari sebuah fungsi yang pada akhirnya dapat “memperkirakan” tinggi badan seseorang berdasarkan masukan umur orang tersebut.
 \item Pembelajaran Tak Terarah (Unsupervised Learning) memodelkan himpunan masukan, seperti penggolongan (clustering).Algoritma ini mempunyai tujuan untuk mempelajari dan mencari pola-pola menarik pada masukan yang diberikan. Meskipun tidak disediakan keluaran yang tepat secara eksplisit. Salah satu algoritma unsupervised learning yang paling umum digunakan adalah clustering/pengelompokan .
       \par Contoh unsupervised learning dalam dunia nyata misalnya  seorang supir taksi yang secara perlahan-lahan menciptakan konsep “macet” dan “tidak macet” tanpa pernah diberikan contoh oleh siapapun .
 \item Pembelajaran Semi Terarah (Semi-supervised Learning)ipe ini menggabungkan antara Supervised dan Unsupervised untuk menghasilkan suatu fungsi.
       \par Algoritma pembelajaran semi terarah menggabungkan kedua tipe algoritma di atas, di mana diberikan contoh masukan-keluaran yang tepat dalam jumlah sedikit dan sekumpulan masukan yang keluarannya belum diketahui. Algoritma ini harus membuat sebuah rangkaian kesatuan antara dua tipe algoritma di atas untuk dapat menutupi kelemahan pada masing-masing algoritma.
       \par Misalnya sebuah sistem yang dapat menebak umur seseorang berdasarkan foto orang tersebut.  Sistem tersebut membutuhkan beberapa contoh, misalnya yang didapatkan dengan mengambil foto seseorang dan menanyakan umurnya (pembelajaran terarah). Akan tetapi, pada kenyataannya beberapa orang sering kali berbohong tentang umur mereka sehingga menimbulkan noise pada data. Oleh karena itu, digunakan juga pembelajaran tak terarah agar dapat saling menutupi kelemahan masing-masing, yaitu noise pada data dan ketiadaan contoh masukan-keluaran.
 \item Reinforcement Learning Tipe ini mengajarkan bagaimana cara bertindak untuk menghadapi suatu masalah, yang suatu tindakan itu mempunyai dampak. 
       \par Adalah sebuah algoritma pembelajaran yang diterapkan pada agen cerdas agar ia dapat menyesuaikan dengan kondisi dilingkungannya, hal ini dicapai dengan cara memaksimalkan nilai dari hadiah ‘reward’ yang dapat dicapai. Suatu hadiah didefinisikan sebuah tanggapan balik ‘feedback’ dari tindakan agen bahwa sesuatu baik terjadi.Sebagai contoh, sangatlah sulit untuk memrogram sebuah agen untuk menerbangkan sebuah helikopter, tetapi dengan memberikan beberapa nilai negatif untuk menabrak, bergoyang-goyang, serta melenceng dari jalur tujuan perlahan-lahan agen tersebut dapat belajar menerbangkan helikopter dengan lebih baik.
\end{enumerate}

\subsection{Contoh Penerapan Machine Learning}
 Contoh penerapan machine learning dalam kehidupan adalah sebagai berikut :
\begin{enumerate}
 \item Penerapan di bidang kedoteran contohnya adalah mendeteksi penyakit seseorang dari gejala yang ada. Contoh lainnya adalah mendeteksi penyakit jantung dari rekaman elektrokardiogram.
 \item Pada bidang computer vision contohnya adalah penerapan pengenalan wajah dan pelabelan wajah seperti pada facebook. Contoh lainnya adalah penterjemahan tulisan tangan menjadi teks.
 \item Pada biang information retrival contohnya adalah penterjemahan bahasa dengan menggunakan komputer, mengubah suara menjadi teks, dan filter email spam.
\end{enumerate}

Salah satu teknik pengaplikasian machine learning adalah supervised learning. Seperti yang dibahas sebelumnya, machine learning tanpa data maka tidak akan bisa bekerja. Oleh karena itu hal yang pertama kali disiapkan adalah data. Data biasanya akan dibagi menjadi 2 kelompok, yaitu data training dan data testing. Data training nantinya akan digunakan untuk melatih algoritma untuk mencari model yang cocok, sementara data testing akan dipakai untuk mengetes dan mengetahui performa model yang didapatkan pada tahapan testing.
\par Dari model yang didapatkan, kita dapat melakukan prediksi yang dibedakan menjadi dua macam, tergantung tipe keluarannya. Jika hasil prediksi bersifat diskrit, maka dinamakan proses klasifikasi. Contohnya klasifikasi jenis kelamin dilihat dari tulisan tangan (output laki dan perempuan). Sementara jika kelurannya bersifat kontinyu, maka dinamakan proses regresi. Contohnya prediksi kisaran harga rumah di kota Bandung (output berupa harga rumah).

\subsection{Dampak Machine Learning di Masyarakat}
Penerapan teknologi machine learning mau tidak mau pasti telah dirasakan sekarang. Setidaknya ada dua dampak yang saling bertolak belakang dari pengembangan teknolgi machine learning. Ya, dampak positif dan dampak negatif.
\par Salah satu dampak positif dari machine learning adalah menjadi peluang bagi para wirausahawan dan praktisi teknologi untuk terus berkarya dalam mengembangkan teknologi machine learning. Terbantunya aktivitas yang harus dilakukan manusia pun menjadi salah satu dampak positif machine learning. Sebagai contohnya adalah adanya fitur pengecekan ejaan untuk tiap bahasa pada Microsoft Word. Pengecekan secara manual akan memakan waktu berhari-hari dan melibatkan banyak tenaga untuk mendapatkan penulisan yang sempurna. Tapi dengan bantuan fitur pengecekan ejaan tersebut, secara real-time kita bisa melihat kesalahan yang terjadi pada saat pengetikan.
\par Akan tetapi disamping itu ada dampak negatif yang harus kita waspadai. Adanya pemotongan tenaga kerja karena pekerjaan telah digantikan oleh alat teknologi machine learning adalah suatu permasalahan yang harus dihadapi. Ditambah dengan ketergantungan terhadap teknologi akan semakin terasa. Manusia akan lebih terlena oleh kemampuan gadget-nya sehingga lupa belajar untuk melakukan suatu aktivitas tanpa bantuan teknologi.

\section{ML on PHP}
\par Beberapa pendekatan yang dapat dimanfaatkan  untuk ML dalam PHP. Misalnya Algoritma, Validasi Silang, Jaringan Saraf Tiruan, Pra-pemrosesan, Ekstraksi Fitur, dan banyak lagi contoh lainnya.

\subsection{Fitur ML on PHP}
Beberapa fitur yang dapat diterapkan dengan ML on PHP:
\begin{enumerate}
\item Association rule Learning
\begin{itemize}
\item Apriori
Pembelajaran aturan asosiasi berdasarkan algoritma Apriori untuk sering melakukan penambangan item.
\begin{itemize}
	\item Constructor Parameters
	\item Train
	\item Predict
	\item Associating
	\item Frequent item sets
\end{itemize}
\end{itemize}
\item Classification
\begin{itemize}
\item SVC
\item k-Nearest Neighbors
\item Naive Bayes
\end{itemize}
\item Regression
\begin{itemize}
\item Least Squares
\item SVR
\end{itemize}
\item Clustering
\begin{itemize}
\item k-Means
\item DBSCAN
\end{itemize}
\item Metric
\begin{itemize}
\item Accuracy
\item Confusion Matrix
\item  Classification Report
\end{itemize}
\item Workflow
\begin{itemize}
\item  Pipeline
\end{itemize}
\item Neural Network
\begin{itemize}
\item Multilayer Perceptron Classifier
\end{itemize}
\item Cross Validation
\begin{itemize}
\item Random Split
\item Stratified Random Split
\end{itemize}
\item Feature Selection
\begin{itemize}
\item Variance Threshold
\item SelectKBest
\end{itemize}
\item Preprocessing
\begin{itemize}
\item Normalization
\item Imputation missing values
\end{itemize}
\item Feature Extraction
\begin{itemize}
\item Token Count Vectorizer
\item Tf-idf Transformer
\end{itemize}
\item Datasets
\begin{itemize}
\item Array
\item CSV
\item Files
\item SVM
\item MNIST
\end{itemize}
\item Models management
\begin{itemize}
\item Persistency
\end{itemize}
\item Math
\begin{itemize}
\item Distance
\item Matrix
\item Set
\item Statistic
\end{itemize}
\end{enumerate}


\subsection{Berkontribusi ke PHP-ML}
\par PHP-ML adalah proyek sumber terbuka. Jika Anda ingin berkontribusi, silakan baca teks berikut ini. Sebelum saya dapat menggabungkan Permintaan Tarik Anda, berikut adalah beberapa panduan yang perlu Anda ikuti. Panduan ini ada untuk tidak mengganggu Anda, tetapi untuk menjaga basis kode tetap bersih, terpadu, dan bukti di masa mendatang.
\begin{enumerate}
\item Cabang
Anda hanya harus membuka permintaan tarik terhadap cabang master.
\item Tes Unit
Coba tambahkan tes untuk permintaan tarik Anda. Anda dapat menjalankan unit-test dengan skript:
	vendor/bin/phpunit
\item Tes Kinerja
Sebelum menjalankan skrip bootstrap pertama kali, akan mengunduh semua set data yang diperlukan dari repositori publik : php-ai/php-m-datasets.
Tes kinerja waktu:
vendor/bin/phpbench run--report=time
Tes kinerja memori:
vendor/bin/phpbench run --report=memory
\item Travis
GitHub secara otomatis menjalankan permintaan tarik Anda melalui Travis CI. Jika Anda melanggar tes, saya tidak dapat menggabungkan kode Anda, jadi pastikan kode Anda berfungsi sebelum membuka Permintaan Tarik.
\item Menggabungkan
Tolong beri saya waktu untuk meninjau permintaan tarik Anda. Saya akan memberikan yang terbaik untuk meninjau semuanya secepat mungkin, tetapi tidak selalu sesuai dengan harapan saya.
\item Standar Pengkodean \& Analisis Statis
Saat berkontribusi kode ke PHP-ML, Anda harus mengikuti standar pengkodeannya. Untuk melakukannya, jalankan:
	composer fix-cs
\end{enumerate}





















\begin{enumerate}
	\item pembelajaran terarah (Supervised Learning)
	\item pembelajaran tak terarah (Unsupervised Learning)
	\item Pembelajaran semi terarah (Semi-supervised Learning)
	\item Reinforcement Learning
\end{enumerate}



\chapter{Studi Kasus dan Penyelesaian}
\section{Perintah Navigasi}
Perintah navigasi direktori


\chapter{Judul Bagian Keenam}
\section{Chapter Baru}
silahkan diisi



\bibliographystyle{IEEEtran} 
%\def\bibfont{\normalsize}
\bibliography{references}


%%%%%%%%%%%%%%%
%%  The default LaTeX Index
%%  Don't need to add any commands before \begin{document}
\printindex

%%%% Making an index
%% 
%% 1. Make index entries, don't leave any spaces so that they
%% will be sorted correctly.
%% 
%% \index{term}
%% \index{term!subterm}
%% \index{term!subterm!subsubterm}
%% 
%% 2. Run LaTeX several times to produce <filename>.idx
%% 
%% 3. On command line, type  makeindx <filename> which
%% will produce <filename>.ind 
%% 
%% 4. Type \printindex to make the index appear in your book.
%% 
%% 5. If you would like to edit <filename>.ind 
%% you may do so. See docs.pdf for more information.
%% 
%%%%%%%%%%%%%%%%%%%%%%%%%%%%%%

%%%%%%%%%%%%%% Making Multiple Indices %%%%%%%%%%%%%%%%
%% 1. 
%% \usepackage{multind}
%% \makeindex{book}
%% \makeindex{authors}
%% \begin{document}
%% 
%% 2.
%% % add index terms to your book, ie,
%% \index{book}{A term to go to the topic index}
%% \index{authors}{Put this author in the author index}
%% 
%% \index{book}{Cows}
%% \index{book}{Cows!Jersey}
%% \index{book}{Cows!Jersey!Brown}
%% 
%% \index{author}{Douglas Adams}
%% \index{author}{Boethius}
%% \index{author}{Mark Twain}
%% 
%% 3. On command line type 
%% makeindex topic 
%% makeindex authors
%% 
%% 4.
%% this is a Wiley command to make the indices print:
%% \multiprintindex{book}{Topic index}
%% \multiprintindex{authors}{Author index}

\end{document}


